\documentclass[11pt]{article}
\date{\today}
\usepackage{fullpage}
\usepackage{fancyhdr}
\usepackage{amsmath}
\usepackage{amssymb}
\usepackage{amsfonts}
\usepackage{amsthm}
\usepackage{bbm}
\usepackage{sectsty}
\usepackage{slashbox}
\usepackage{rotating}
\usepackage[colorlinks]{hyperref}
\hypersetup{
pdfauthor={Christopher M. Maes},
citecolor=black,
linkcolor=black,
urlcolor=black
}

\newfont{\sansf}{cmss10}
\newfont{\solverf}{cmss10 scaled 1100}
\newfont{\bssten}{cmssbx10}
\newfont{\bssnine}{cmssbx10 scaled 900}
\newfont{\bssdoz}{cmssbx10 scaled 1200}


\newcommand{\type}{Notes for camera projection}
\newcommand{\info}{Christopher Maes}
\pagestyle{fancy} 
\date{\today}
\chead{\type}
\rhead{\today}
\lhead{\info}
\rfoot{}
\cfoot{\thepage}
\lfoot{}
\renewcommand{\footrulewidth}{.4pt}
\addtolength{\headsep}{.5in}
\renewcommand{\headheight}{14pt}

\setlength{\parskip}{1ex plus 0.5ex minus 0.2ex}
\setlength{\parindent}{0in}

\newcommand{\norm}[1]{\left\lVert\,#1\,\right\rVert}
\newcommand{\rank}{\mathrm{rank}}
\newcommand{\pmat}[1]{\begin{pmatrix}#1\end{pmatrix}}
\newcommand{\abs}[1]{\left\vert #1 \right\vert}
\newcommand{\set}[2]{\left\{#1  \left| #2\right\}\right.}
\newcommand{\R}{\mathbbm{R}}
\newcommand{\matlab}{{\sc Matlab}}
\newcommand{\dims}[2]{\R^{#1 \times #2}}
\newcommand{\diag}{\mathrm{diag}}
\newcommand{\range}{\mathcal{R}}
\newcommand{\nullspace}{\mathcal{N}}

\newcommand{\maximize}{\text{maximize }}
\newcommand{\minimize}{\text{minimize }}
\newcommand{\st}{\text{subject to }}
\renewcommand{\l}{\ell}
\newcommand{\half}{{\textstyle{\frac12}}}



\begin{document} 


The main mathematical problem associated with the camera projection 
problem is: 

Given the projections of a point $x \in \R^n$ onto a set of $m$ 
$n-1$ dimensional hyperplanes, how do we reconstruct an estimate of
$x$?

Suppose the hyperplanes are defined as 
\[
\mathcal{P} = \{ x \in \R^n \ |\ a^T x = \beta  \}.
\] 
Let the columns of the matrix $Z \in \dims{n}{n-1}$ form a basis for
the nullspace of $a^T$. That is, $a^T Z = 0$.  And suppose we are
given $x_0 \in \R^n$ which defines an origin of the hyperplane with
$a^T x_0 = \beta$, and $y \in \R^{n-1}$ the $n-1$ dimensional
projection of $x$ onto the hyperplane.

Define $u = x - x_0$, and let $P u$ be the projection of $u$ onto the
hyperplane, then $Pu = Z y$. We may write $u = w + v$ where $w \in
\range(Z)$ and $v \in \nullspace(Z^T)$.  So that
\[ 
u = w + v = Z y + v 
\] 
Since we have that $Z^T v = 0$, multiplying the above by $Z^T$ we have
$Z^T u = Z^T Z y$, or $y = (Z^T Z)^{-1} Z^T u$.  So the projection
matrix $P$ is defined as $P = Z (Z^T Z)^{-1} Z^T$. Thus, we have
that 
\[
P (x - x0) = Z y,
\]
where $P$, $x_0$, $y$, and $Z$ are defined from the data, and $x$ 
is unknown. 

Let $P_i$ and $Z_i$ denote the projection and nullspace matrix 
associated with the $i$th hyperplane, and $x_0^i$ and $y_i$ 
denote the $i$th hyperplane origin and projected coordinate. 
To determine an estimate for $x$ we may solve the least-squares
problem 
\begin{equation}
\underset{x}{\minimize}  \quad  \sum_{i=1}^m \norm{ P_i x - P_i x_0^i - Z_i y_i} 
\end{equation}
This is equivalent to the problem 
\[
\underset{x}{\minimize} \quad \norm{ \pmat{P_1 \\ \vdots \\ P_m } x 
                                    -\pmat{P_1 x_0^1 - Z_1 y_1 \\ \vdots \\ P_m x_0^m - Z_m y_m} }
\]




    
\end{document}
